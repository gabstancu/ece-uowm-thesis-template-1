\begin{codeblock}{python}
def hello(name):
    print("Hello", name)
\end{codeblock}

\begin{codeblock}{c++}
#ifndef SOR_HPP
#define SOR_HPP

#include "utils/SolverLog.hpp"
#include "solvers/config.h"
template<typename Vector>
struct SOR
{   
    using Scalar       = typename Vector::Scalar;
    using SparseMatrix = Eigen::SparseMatrix<Scalar, Eigen::RowMajor>;

    double              tol       = DEFAULT_TOL;
    int                 max_iters = MAX_ITERS;
    double              omega;
    std::string         name      = "SOR";
    SolverLog<Vector>   log;
    Vector              final_solution;
    
    template<typename System>
    SOR (System system) : omega(system.omega_)
    {   
        log.system_dim     = system.A.rows();
        max_iters          = static_cast<int>(10 * std::sqrt(log.system_dim));
        log.max_iterations = max_iters;
        log.tolerance      = tol;
        log.solver_name    = name;
    }

    template<typename System>
    void solve(System& system)
    {   
        const auto& A        = system.A;
        const auto& b        = system.b;
              auto& u        = system.u;

        std::cout << "max_iters: " << max_iters << '\n';

        double sum1, sum2;
        
        double b_norm   = b.norm();
        double res      = (A * u - b).norm() / b_norm;

        if (res <= tol) 
        {   
            this->final_solution = u;
            log.final_solution   = this->final_solution;
            log.converged = 1;
            return;
        }

        Vector inv_diag = A.diagonal().cwiseInverse();

        for (int k = 0; k < max_iters; k++)
        {   
            for (int i = 0; i < A.rows(); ++i)
            {   
                double sum = 0;

                for (typename SparseMatrix::InnerIterator it(A, i); it; ++it)
                {
                    int j = it.col();

                    if (j != i)
                    {
                        sum += it.value() * u[j];
                    }
                }
                u[i] = (1 - omega) * u[i] + omega * (inv_diag[i] * (b[i] - sum));
            }
             
            res = (A * u - b).norm() / b.norm();

            log.num_of_iterations++;
            log.res_per_iteration.push_back(res);

            if (res <= tol) 
            {   
                this->final_solution = u;
                log.final_solution   = this->final_solution;
                log.converged = 1;
                return;
            }
        }
        this->final_solution = u;
        log.final_solution   = this->final_solution;
        return;
    }
};


#endif // SOR_HPP
\end{codeblock}

\begin{codeblock}{C++}
#ifndef CONJUGATE_GRADIENT_HPP
#define CONJUGATE_GRADIENT_HPP

#include "utils/SolverLog.hpp"
#include "solvers/config.h"
template< typename Vector>
struct ConjugateGradient
{
    double              tol       = DEFAULT_TOL;
    int                 max_iters = 1e6;
    std::string         name      = "CG";
    SolverLog<Vector>   log;
    Vector              final_solution;

    ConjugateGradient ()
    {
        log.tolerance      = tol;
        log.max_iterations = max_iters;
        log.solver_name    = name;
    }

    template<typename System>
    void solve(System& system)
    {   
        const auto&  A = system.A;
        const auto&  b = system.b;
        auto&        u = system.u;
        log.system_dim = A.rows();

        std::cout << "max_iters: " << max_iters << '\n';

        Vector r      = b - A * u; // initial residual
        double b_norm = b.norm();
        double r_norm = r.norm();

        if (r_norm / b_norm <= tol) 
        {   
            this->final_solution = u;
            log.final_solution   = this->final_solution;
            log.converged        = 1;
            return;
        }

        Vector d = r; // initial search direction
        Vector Ad(A.rows());

        for (int k = 0; k < max_iters; k++)
        {   
            // std::cout << "--------------------- iter. " << k+1 << " ---------------------\n";
            Ad.noalias() = A * d;

            double alpha      = ((r.transpose() * r) / (d.transpose() * Ad)).coeff(0); // step size
            double r_prev_dot = (r.transpose() * r).coeff(0); // to calculate beta

            u.noalias() += alpha * d;
            r.noalias() -= alpha * Ad;

            r_norm = r.norm();
            
            log.num_of_iterations++;
            log.res_per_iteration.push_back(r_norm / b_norm);

            if (r_norm / b_norm <= tol) 
            {   
                log.converged        = 1;
                this->final_solution = u;
                log.final_solution   = this->final_solution;
                return;
            }

            double beta = r.dot(r) / r_prev_dot;
            d           = r + beta * d; // update direction
        }
        this->final_solution = u;
        log.final_solution   = this->final_solution;
        return;
    }
};

#endif // CONJUGATE_GRADIENT_HPP
\end{codeblock}